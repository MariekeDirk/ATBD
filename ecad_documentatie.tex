\documentclass{article}

\bibliographystyle{abbrv}

%-- wider textwidth:
%\setlength{\leftmargin}{-5mm}
%\setlength{\rightmargin}{\leftmargin}
%\setlength{\oddsidemargin}{\leftmargin}
%\setlength{\evensidemargin}{\leftmargin}
%\setlength{\textwidth}{170mm}

%-- wider textheight:
%\setlength{\textheight}{240mm}
%\setlength{\topmargin}{-10mm}

\newcommand{\dF}{$^{\circ}$F}
\newcommand{\dC}{$^{\circ}$C}
\newcommand{\dN}{$^{\circ}$N}
\newcommand{\dS}{$^{\circ}$S}
\newcommand{\dW}{$^{\circ}$W}
\newcommand{\dE}{$^{\circ}$E}

\newcommand{\veps}{\varepsilon}
\newcommand{\rd}{\text{d}}


\usepackage{amsmath}
\usepackage{amssymb}
\usepackage{epsfig}
\usepackage{makeidx}
\usepackage{rotating}
\usepackage{bezier}
\usepackage{tabularx}

\begin{document}

\section*{ECA\&D Documentatie}

\subsection*{ECA\&D ontwikkel- en testomgeving}
De ECA\&D ontwikkel- en testomgeving is de bhlbontw.knmi.nl. Er zijn twee gebruikersaccounts op deze machine: 
rccuser en rcctest. Beide hebben als password: welkom\\

\noindent
Home directory: /data/apps/rcc\\
directory met webpagina's: /data/apps/rcc/codebase\\
directory met TriOpSys' processing controller: /data/apps/rcc/server\\

\noindent
Dit systeem is niet toegankelijk voor de buitenwereld, geblokkerd door de firewall. De webpagina's zijn intern
te benaderen via: http://bhlbontw/rcc\\

\noindent
Op de bhlbontw is een mysql-server (localhost) aanwezig.
De toegangen is:
\begin{itemize}
\item user: rccuser password: rcc123
\end{itemize}

\subsection*{ECA\&D productieomgeving}
De ECA\&D productieomgeving os een cluster van de machines bhlbecad01/02 en is te benaderen via: ecad.knmi.nl.
Toegang via user: rccuser password: 3cAtUsr.

\noindent
De inrichting van de directory- en filestructuur is gelijk gehouden aan de test- en ontwikkelomgeving. Het is de 
bedoeling dat de databases op de beide machines ook indentiek zijn.

Op dit systeem is geen mysql-server, maar er is toegang tot een extrene mysql-server. 
\begin{itemize}
\item power user: user: powerecad password: pwr3cAt
\item read \& write: user: 3cAtr8 password: rwecad
\item read-only: user: ecadread password: 3cAtrd
\item databasenaam: RCC
\end{itemize}

\noindent
Je kunt de database benaderen door bijvoorbeeld:\\
\texttt{mysql -u powerecad -p -h bwdb}\\
waarna er om het password gevraagd wordt.

\noindent
Het vullen van tabellen kan je doen door:\\
\texttt{mysql -hbwdb.knmi.nl -upowerecad -ppwr3cAt RCC < tabelnaam.sql}

\subsection*{Nieuwe releases van TriOpSys}
De server applicatie verwijst ook naar het Server.jar bestand (die middels Apache gehost is). Hierbij dienen 
deze bestanden dan wel exact hetzelfde te zijn, anders treden er conflicten op in de verbinding met RMI. 
De melding die zicht hieronder voordoet kan het volgende als oorzaak hebben, namelijk:
\begin{itemize}
\item dat de versie van de bestanden afwijken van elkaar (Apache en de server applicatie)
\item dat de server applicatie geen toegang heeft tot het bestand (een verkeerde verwijzing, rechten problemen 
of Apache die nog niet gestart is).
\end{itemize}

Hierom zou het het mooiste zou zijn als de installatie als volgt zou zijn:

Server applicatie:
\begin{itemize}
\item   /data/apps/rcc/server/startServer.sh
\item   /data/apps/rcc/server/Server.jar $\rightarrow$ /data/apps/rcc/codebase/Server$_-$1074.jar (symbolic link)
\item   /data/apps/rcc/server/config/rcc.properties
\item   /data/apps/rcc/server/config/security.policy
\item   Etc.
\end{itemize}

Codebase:
\begin{itemize}
\item  /data/apps/rcc/codebase/Server$_-$1074.jar
\end{itemize}

Apache HTTP server:
\begin{itemize}
\item  /var/www/rcc / Server.jar $\rightarrow$ /data/apps/rcc/codebase/Server$_-$1074.jar (symbolic link)
\end{itemize}

Door het Server$_-$xxxx.jar bestand onder de codebase directory te plaatsen en gebruik te maken van 
twee symbolic links kunnen deze bestanden nooit uit verschillende versies bestaan. Daarnaast hoeft er dan maar 
eenmalig in de security.policy bestanden van de server, console en web applicatie een verwijzing naar het 
Server.jar bestand gemaakt te worden, die dan middels de Apache web server beschikbaar wordt gesteld.

Let op dat er ook een verwijzing wordt gedaan naar het RMI Server.jar bestand in het startServer.sh script. 
Deze dient ook aangepast te worden en indien bovengenoemde configuratie wordt gehanteerd zal ook deze 
aanpassing eenmalig zal zijn.

Voor het plaatsen van de nieuwe web applicatie hoeft deze in principe niet opnieuw opgestart te worden 
echter is het wel verstandig om dit te doen aangezien je er niet geheel zeker van kan zijn dat alles opnieuw 
ingeladen wordt. Daarnaast loop je ook het risico dat de gebruiker mogelijk de verkeerde url kan gaan gebruiken.

\subsection*{invoeren van data van participanten}
Voor een aantal participanten is alle of een deel van de data in te voeren vanaf de website van het betreffende meteorologische
instituut, of (in het geval van NL), vanuit de KIS database. Om de juiste koppeling tussen de stationsnamen/stations id's van elk 
land en de ECA\&D sta\_id's te waarborgen, wordt een tabel aangemaakt waarin deze koppeling expliciet gemaakt wordt.

In die tabel staan achtereenvolgens:\\
stationsnaam\\ 
provincie (optioneel)\\
nationaal stations id \\
ECA\&D sta\_id\\
cc ser\_id \\
ss ser\_id \\
pp ser\_id \\
rr ser\_id \\
sd ser\_id \\
tg ser\_id \\
tn ser\_id \\
tx ser\_id \\
hu ser\_id \\
hierbij wordt een ser\_id op "-1" gezet als het betreffende element voor dat station mist.

\subsubsection*{invoeren Nederlandse data}
De invoer van Nederlandse data wordt gedaan met een bash script wat direct verbinding zoekt met het Klimatologisch 
Informatie Systeem,
uurlijkse data uit de database haalt, en deze aggregeert naar dagelijkse waarden. Dit aggregeren gebeurt door drie Fortran routines.

Aanroep van het script is:
\texttt{sh update\_Netherlands.sh} yyyymmdd yyyymmdd\\
waarbij yyyymmdd the start- en einddatum zijn respectivelijk. Dit file produceert een aantal datafiles, met als naam: 
\texttt{Stationsnaam\_XX.txt}. Hierbij is XX $\in$ (TX,TG,TN,PP,RH,SS,CC,RR,SD). In deze files staat: datum, ser\_id, ele\_name, 
value. 

Het bash script staat vooralsnog op: /net/bhw132/nobackup\_1/users/schrier/ECAD/jobs.

\subsubsection*{invoeren data van Luxembourg Airport}
Maandelijks stuurt Jacques Zimmer een Excel file met de data van de afgelopen maand. Dit Excel file (naam: "ARRAY10 YYMM.xls") 
wordt locaal gesaved en omgezet naar een csv file. Dit gebeurt op de command line met een macro:\\
\texttt{OOo-calc -invisible ARRAY10\ 0109.xls "macro:///Standard.Module1.To\_ascii"}\\
Dit geeft het file to\_ascii.csv.

Met het script: \texttt{update\_Luxembourg.sh} is de data in het juiste format te krijgen voor invoer. Hierbij wordt \emph{geen}
gebruik gemaakt van een extern file wat nationaal stations id koppelt aan ECA\&D stations id.

Het bash script staat vooralsnog op: /net/bhw132/nobackup\_1/users/schrier/ECAD/jobs.

\subsection*{invoeren van SYNOP data van het GTS netwerk}
Synop data wordt per maand vanaf het ECMWF binnen gehaald. Log in bij het ECMWF: \texttt{ssh -l nk6 -X ecaccess.ecmwf.int} en kies
directory: ecgate. Login via de "Acividentity" tool, pin code: 0778. 
In dit directory staat het script: runall.sh. Dit script dient handmatig aangepast te worden om data van de afgelopen maand binnen 
te halen. Deze data wordt verzonden naar de ecadev machine (moet nog veranderd worden naar de bhlbontw).

Het script: apps/scripts/meta\_dailyscripts.sh pakt de bufr data uit tot een ASCII bestand en vult de tabel \texttt{synops} 
met de data.  Vanuit de tabel \texttt{synops} wordt de tabel \texttt{synops\_yy} gevuld.




\subsection*{Backups op het MOS}
Er bestaat een account 'rccuser' op het Massa Opslag Systeem voor backups van het ECA\&D systeem. 
De UID op het MOS is 7205, de group is 'ksarch', dus daarmee kun je het KS archief in. Let wel: bij inloggen kom je binnen in 
/usr/people/rccuser, waar je verder niets kunt opslaan. Het directory waar je wel iets kunt opslaan is: /fa/ks/RCC.
Het password is '3c4dRCC'. 

In de directory /data/apps/rcc/scripts op de bhlbontw staan drie scripts die een backup van de database, van de website en van de scripts
maken en deze op het MOS zetten. De naam van de directory waar deze backups onder staan in gekoppeld aan de dag waarop de backup gemaakt
wordt. Er is geen functionaliteit die oude backup weggooit.

\subsection*{Permissies}







\bibliography{bib/refer}


\end{document}
