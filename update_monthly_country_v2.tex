\documentclass[a4paper]{article}

\title{Monthly updates for Norway, Slovenia, Germany, Finland, the
  Netherlands, the Czech Republic, Luxemburg, Wageningen-Haarweg,
  and how to make the monthly E-OBS files}

\author{Else van den Besselaar \& Gerard van der Schrier}

\begin{document}

\maketitle


\section{Introduction}
This document describes how the monthly updates for Norway, Slovenia,
Germany, Finland, the Netherlands, the Czech Republic, Luxemburg, and
Wageningen-Haarweg are included into the ECA\&D database.

%% NORWAY:

\section{Norway}

The Norwegian updates will be received in the morning of the first day
of the month via email at eca@knmi.nl. The data includes days for the
last three months. All steps are relative to the directory
\texttt{$\sim$/Else/Countries/Norway/AllElements} on the ECA\&D development system
bhlbontw.

\subsection*{Steps to take}

New reports have been ordered from Eklima (met.no) so now all stations
reports have the same format:
\begin{enumerate}
\item Check directory \texttt{Data} for datafiles. If they exist, move
  them to a directory \textit{yyyymm} with \textit{yyyy} year and
  \textit{mm} the month of the previous update.
\item Copy the files from the emails to \texttt{Data}.
\item Typing './run\_updates\_allelements.sh' should be enough.
\item The script should copy all datafiles to the directory
  \textit{Data/yyyymm} with \textit{yyyy} year and \textit{mm} the
  previous month.
\item If problems arise, probably something has changed in the format
  of the data. Changing the program
  \texttt{update\_norway\_allelements.c} and using 'make
  update\_norway\_allelements' might be needed. (Not happened so far.)
\end{enumerate}


%% SLOVENIA

\section{Slovenia}

Conny Schiks downloads the data from the website
\texttt{http://meteo.arso.gov.si/ met/en/app/webmet/} (icon
Archive). She sends an email to \texttt{eca@knmi.nl} with information
on where to find the data. All steps are relative to the directory
\texttt{$\sim$/Else/Countries/Slovenia} on the ECA\&D development
system bhlbontw.

\subsection*{Steps to take}

\begin{enumerate}
\item Check for datafiles in the directories \texttt{Datacumulative},
  \texttt{Datameans} and \texttt{Dataprecip}. If there are datafiles
  present, move them to a directory \textit{yyyymm} with \textit{yyyy}
  year and \textit{mm} the month of the previous update.
\item Copy the datafiles from the directories including 'cumulatives'
  of Conny's email to \texttt{Datacumulative/}
\item Copy the datafiles from directories including 'Measurements and
  Means' of Conny's email to \texttt{Datameans/}
\item Copy the datafiles from the directories including
  'precipitation' of Conny's email to \texttt{Dataprecip/}
\item Make sure that in \texttt{update\_slovenia\_cumulative.c} lines
  394 to 416 are commented out (with // in front of the line or /*
  \dots */ (this is the part that actually includes the data into the
  database).
\item Type 'ls Datacumulative/*.txt $>$ inputdatacumulative.txt'
\item Type 'ls Datameans/*.txt $>$ inputdatameans.txt'
\item Type 'make update\_slovenia\_cumulative'
\item Type 'update\_slovenia\_cumulative'
  \begin{itemize}
  \item Most likely some errors are produced saying that certain
    stations don't exist in \texttt{series\_slovenia.txt}. Usually
    this means that Conny made a typing error in the station name
    which is part of the filename. Check \texttt{series\_slovenia.txt}
    for the correct name (ignore uppercase/lowercase
    differences). Also the first line of the datafile gives the
    station name. Change the filename to the correct name and also in
    \texttt{inputdatacumulative.txt} (please do not change
    \texttt{series\_slovenia.txt} as that might give problems when,
    for some reason, we need to include earlier data again).
  \item Other errors might come from something in the datafile. For
    example some commas are missing somewhere in the file due to the
    copy-paste work from Conny. Include the missing commas and it
    should work. As of March 2013, Conny no longer includes comma's
    since the Slovenian website has changed. She will use Excell and
    save as csv which gives ; separated files. The programs changes
    these to comma's, no need to do it yourself.
  \item It could also be that the filename does not end with
    \texttt{\#\#\#\#-\#\#\#\#.txt}. If this is the case, include
    \textit{monthyear-monthyear} so that the format is correct (values
    itself are not used, Conny uses them to see what she has already
    downloaded).
  \item If a certain datafile keeps giving problems that you can not
    solve, put a \texttt{\#} in front if it in
    \texttt{inputdatacumulative.txt} (than it will be skipped) and put
    the datafile somewhere separately. I can have a look at it later.
  \end{itemize}
\item Type 'update\_slovenia\_cumulative' again until no more errors
  are seen.
\item Remove the // or /* */ in
  \texttt{update\_slovenia\_cumulative.c} in front of the include
  parts and type 'make update\_slovenia\_cumulative'
\item Type 'update\_slovenia\_cumulative' again to really include the
  data into the database.
\item Move the datafiles to a new directory \textit{yyyymm} in
  \texttt{Datacumulative/} with \textit{yyyy} year and \textit{mm} the
  month of this update.
\end{enumerate}

Repeat steps 5 to 13 for \texttt{inputdatameans} and
\texttt{update\_slovenia\_means.c} (lines 439 to 494 need to be
commented out at first) and also for \texttt{inputdataprecip.txt} and
\texttt{update\_slovenia\_precip} (lines 397 to 411 need to be
commented out at first).


%% GERMANY

\section{Germany}

The German data will be put on the ftp-server \texttt{ftppro.knmi.nl}
using user \textit{eca} in the directory \texttt{Germany}. An email at
eca@knmi.nl is received when the data is available from the
ftp-server, usually around the second of the month. The German data is
updated to the end of the month before last (e.g. they are always one
month behind our updates). All steps are relative to the directory
\texttt{$\sim$/Else/Countries/Germany/FTPserver/} on the ECA\&D
development system bhlbontw.

\subsection*{Steps to take}

\begin{enumerate}
\item Check if there are no tar-files present in \texttt{Data/Klima}
  and \texttt{Data/Synop}. If there are, move them to
  \texttt{Data/Klima/Done} and \texttt{Data/Synop/Done}.
\item Copy the tar-file with 'klima' in the name from the ftp-server
  to \texttt{Data/Klima}.
\item Copy the tar-file with 'synop' in the name from the ftp-server
  to \texttt{Data/Synop} (this file contains pressure and maximum of
  hourly precipitation).
\item Type './run\_updates\_ftp.sh'. This takes care of the untarring
  and ungzipping of both the synop and klima data and includes the
  data into the database. It will also move the tar-files to the
  \texttt{Done} directories and removes unnecessary files.
\item If errors occur then it is most likely that one or more of the
  datafiles have an error in (format of) the datafile.
\end{enumerate}


%% Finland

\section{Finland}

The Finnish data will be put on the ftp-server \texttt{ftppro.knmi.nl}
using user \textit{eca} in the directory \texttt{Finland}. The data is
sent on the first day of the month containing data of the last 2
monhts. All steps are relative to the directory
\texttt{$\sim$/Else/Countries/Finland/} on the ECA\&D development
system bhlbontw.


\subsection*{Steps to take}

\begin{enumerate}
\item Check if there are no .dat-files in \texttt{Data}. If there are,
  move them to \texttt{Data/Done}.
\item Copy the .dat-file from the ftp-server to \texttt{Data}.
\item Type './run\_updates\_fmi.sh'. This takes care of including the
  data into the database. It will also move the .dat-file to the
  directory \texttt{Data/Done}.
\end{enumerate}


%% The Netherlands

\section{the Netherlands}

The Netherlands data is extracted from the KNMI KIS-system. In the directory
\texttt{/nobackup/users/schrier/ECAD/jobs} are the scripts needed to make contact
with the KIS database, which extract the relevant data and aggregate the data
if neccessary. These scripts are: \texttt{update\_NL\_rrsd.sh} , \texttt{update\_NL\_txtgtnsspphucc.sh}
and \texttt{update\_NL\_fgfxdd.sh}.

\subsection*{Steps to take}

\begin{enumerate}
\item change to directory \texttt{/nobackup/users/schrier/ECAD/jobs} on the \texttt{bhw323}
\item run the scripts, with input on the command line: the startdate and the enddate for which
you want to add data. Format of these dates: yyyymmdd. 

Note that it takes time for the KNMI
validation to validate all the precipitation (and snow depth) stations. This means that the startdate for
an update of a given month should be \emph{the first day of the previous month}. So, an update
for precipitation for the month November 2011 will have as input to the script: 20111001 20111130.

Note: the precipitation and snowdepth readings of the KNMI network are related to the date of the observing
day, while the measurement overlaps mostly with the day preceding the observation day. In the software,
this difficulty is adjusted for by shifting the date of the measurement one day back. This means
that the enddate for the update script should be the first day of the current month. Example: for November 2011,
this should be 20111201. However, the validation usually lags some 20 days, updating very recent data
need not take account of this issue.
\item in the directory \texttt{../results}, tarred files appear which can be ftp-ed to the \texttt{bhlbontw},
directory \texttt{Gerard/datainput}. Run the script \texttt{insert\_data\_in\_} \texttt{database.sh} 
with Netherlands
on the command line. This script unpacks the data, inserts it into the database and stores the raw
data in the directory \texttt{Gerard/Netherlands/Data/yyyymmdd}, where yyyymmdd the date of the day the insert
script was run.
\end{enumerate}


%% Czech Republic

\section{Czech Republic}

The Czech data will be put on the ftp-server \texttt{ftppro.knmi.nl}
using user \textit{eca} in the directory \texttt{CzechRepublic}. 
The Czech data is updated to the end of the month before last (they are always one
month behind our updates). 
In the directory
\texttt{/nobackup/users/schrier/ECAD/jobs} is the script which translates the data to
a  format ready for ECA\&D.
This script is: \texttt{update\_CzechRepublic.sh}.

\subsection*{Steps to take}

\begin{enumerate}
\item ftp the Czech data to \texttt{/usr/people/schrier/ECAD/participantdata/} \texttt{czechrepublic/}.
\item change to directory \texttt{/nobackup/users/schrier/ECAD/jobs} on the \texttt{bhw323}
\item run the script, with input on the command line: the year and month to be updated (format: yyyymm).

The script will produce a simple txt file in the directory \texttt{../results} 
with data ready to be inserted into the database.
\item ftp the resulting datafile to the \texttt{bhlbontw},
directory \texttt{Gerard/datainput}. Run the script \texttt{insert\_data\_in\_database.sh} with CzechRepublic
on the command line. This script inserts the data into the database and stores the raw
data in the directory \texttt{Gerard/CzechRepublic/Data/yyyymmdd}, where yyyymmdd the date of the day the insert
script was run.
\end{enumerate}


%% Luxemburg

\section{Luxemburg}

The Luxemburg data is send by email early in the new month, usually the first or second of the month.
In the directory
\texttt{/nobackup/users/schrier/ECAD/jobs} is the script which translates the data to
a  format ready for ECA\&D.
This script is: \texttt{update\_Luxembourg.sh}.

\subsection*{Steps to take}

\begin{enumerate}
\item save the attachment in the directory \texttt{/usr/people/schrier/ECAD/parti-} \texttt{cipantdata/luxembourg/}.
\item the attachement is an Excell file, convert this by running soffice and save as CSV file. This produces
a file with the same name as the original file, but with extension .csv.
\item change to directory \texttt{/nobackup/users/schrier/ECAD/jobs} on the \texttt{bhw323}
\item edit the script \texttt{update\_Luxembourg.sh} and change the year and month to the year and month of
the data provided.  
\item run the script \texttt{update\_Luxembourg.sh}. This produces the file \texttt{lux.tar} on \texttt{../results} with data ready to be inserted into the database.
\item ftp the resulting datafile to the \texttt{bhlbontw},
directory \texttt{Gerard/datainput}. Run the script \texttt{insert\_data\_in\_database.sh} with Luxembourg
on the command line. This script inserts the data into the database and stores the raw
data in the directory \texttt{Gerard/Luxembourg/Data/yyyymmdd}, where yyyymmdd the date of the day the insert
script was run.
\end{enumerate}

%% Wageningen - Haarweg

\section{Wageningen - Haarweg}

A special case is the data from Wageningen University (station `Haarweg'). This station used to be a
KNMI `termijn' station (part of an extensive network of climatological stations measuring various
elements at 3-hourly resolution) but is continued under the supervision of the Wageningen agricultural
university (WUR).

The Wageningen data is made available via the WUR webpages and is available some days after the start
of the new month.
In the directory
\texttt{/nobackup/users/schrier/ECAD/jobs} is the script which translates the data to
a  format ready for ECA\&D.
This script is: \texttt{update\_Wageningen.sh}.

\subsection*{Steps to take}

\begin{enumerate}
\item run the script \texttt{update\_Wageningen.sh}. This script requires two input fields: year and month
as yyyy mm and extracts the appropriate datafile from the WUR webpages and processes its contents. 
This produces the file \texttt{Wageningen.tar} on \texttt{../results} with data ready to be 
inserted into the database.
\item ftp the resulting datafile to the \texttt{bhlbontw},
directory \texttt{Gerard/datainput}. Run the script \texttt{insert\_data\_in\_database.sh} with Wageningen
on the command line. This script inserts the data into the database and stores the raw
data in the directory \texttt{Gerard/Wageningen/Data/yyyymmdd}, where yyyymmdd the date of the day the insert
script was run.
\end{enumerate}

%% Spain

% \section{Spain}
% The Spanish data is made available via the ftp server of AEMET. The updates of the daily data are in a compressed
% ASCII file which contains all data from the running year, e.g. 2012.CSV.gz. Spanish data is updated to the end 
% of the previous month

% In the directory
% \texttt{/nobackup/users/schrier/ECAD/jobs} is the script which translates the data to
% a  format ready for ECA\&D.
% This script is: \texttt{update\_Spain.sh}.

% \subsection*{Steps to take}

% \begin{enumerate}
% \item run the script \texttt{update\_Spain.sh}. This script requires one input field: year 
% as yyyy and extracts the appropriate datafile from the AEMET ftp webpages and processes its contents. 
% This produces the file \texttt{Spain.tar} on \texttt{../results} with data ready to be 
% inserted into the database.
% \item ftp the resulting datafile to the \texttt{bhlbontw},
% directory \texttt{Gerard/datainput}. Run the script \texttt{insert\_data\_in\_database.sh} with Spain
% on the command line. This script inserts the data into the database and stores the raw
% data in the directory \texttt{Gerard/Spain/Data/yyyymmdd}, where yyyymmdd the date of the day the insert
% script was run.
% \end{enumerate}


%% E-OBS


\section{Monthly E-OBS update}

Below are the steps needed to produce the monthly update for the E-OBS
dataset. This can be started when \texttt{all\_qcblend\_eobs.sh} in
the update cycle (\texttt{fullcycle.sh} on the development system
(bhlbontw) has been finished. It will give a message: \textit{Data for
  E-OBS is ready!}.

Furthermore, the steps to produce the related plots for the climate
diagnostics talk as well as for the website Maandoverzicht Wereldweer
are given below.


\subsection{Updating the gridded files}

\begin{enumerate}
\item Go to the directory \texttt{Else/Ensembles} on the development
system bhlbontw.
\item Check \texttt{make\_ensembles\_else.sh}: uncomment (if needed)
  only the parts for ensembles\_tx?, ensembles\_tn?, ensembles\_tg?,
  ensembles\_rr? and ensembles\_pp?. The parts with
  ensembles\_t?\_stations, ensembles\_rr\_stations and
  ensembles\_pp\_stations are not needed. What IS needed is the part
  with \texttt{ensembles\_all\_stations}.
\item Type 'make\_ensembles\_else.sh'. The files will be put into the
  directory \texttt{data}.
\item Copy the files
  \texttt{$\sim$/Else/Ensembles/data/ensembles\_t?},
  \texttt{$\sim$/Else/Ensembles/ data/ensembles\_pp},
  \texttt{$\sim$/Else/Ensembles/ data/ensembles\_rr}, and 
  \texttt{$\sim$/Else/ Ensembles/data/ensembles\_all\_stations}
  (located on the development system bhlbontw) to
  \texttt{/nobackup/users/schrier/E-OBS/data\_input} on
  the bhw323 machine (Gerards workstation).\\

\smallskip 

\noindent On the bhw323 machine:
\item Remove the files in
\texttt{/nobackup/users/schrier/E-OBS/gridded\_daily/}
and \linebreak
\texttt{/nobackup/users/schrier/E-OBS/gridded\_climatology/}.
\item In \texttt{/nobackup/users/schrier/E-OBS/final/} remove only the
  files starting with tn$|$tg$|$tx$|$pp$|$rr AND ending with nc or
  nc.gz (the other ones need to stay there!)
\item Check number of stations with "select count(sta\_id from
  stations)" in mysql on the development system bhlbontw.
\item Go to \texttt{/nobackup/users/schrier/E-OBS/programs/final\_gridding} (on the
  bhw323 machine) and change the number of stations at the top of
  grid.inc (nstns) to the one from the query above. Check also the
  directories \texttt{data\_dir}, \texttt{clim\_dir},
  \texttt{daily\_dir} and \texttt{final\_dir} if these are the same as
  the ones from the steps above. Also check the elements that are
  going to be gridded. These are given with \texttt{var} and
  \texttt{dovar} in \texttt{grid.inc}. \texttt{var} only shows the
  other of the elements (do not change this!), \texttt{dovar} shows
  what elements to grid.
\item If only one year needs to be gridding, only the main grid.inc
  needs to be changed. If all years need to be gridding, change also
  the grid.inc files in the directories Do\#\#\#\#\#\# (nstns, dovar
  and directories) as these do the gridding for 10 years each. Then
  also change do\_gridding.sh so that it uses the Do\#\#\#\#\#\#
  directories instead of the grid\_daily\_trim\_else in the
  final\_gridding directory.
\item Type 'do\_gridding.sh' in
  \texttt{/nobackup/users/schrier/E-OBS/programs/final\_gridding} on
  the bhw323 machine. You can also run it via a nohup option if
  preferred.
\item If errors occur during \texttt{ascii\_to\_uf} (quite quickly
  after starting the gridding procedure) it usually means that the
  number of stations in grid.inc is not correct and/or the file
  \texttt{/nobackup/users/schrier/E-OBS/data\_input/\linebreak ensembles\_all\_stations}
  is not updated.
\item If errors occure later on, it probably means that some files
  were not removed from the \texttt{gridded\_daily} or
  \texttt{gridded\_climatology} directories mentioned in some steps
  before and therefore the program can not update/write the files it
  is needed.
\item After \texttt{do\_gridding.sh} is finished without errors, copy
  the files in \texttt{/nobackup/\linebreak users/schrier/E-OBS/final/} to
  directory \texttt{$\sim$/Gridding/Data/\$year\$/final} on the bhle4m
  machine.
\item run \texttt{makemonths.sh} in
  \texttt{$\sim$/Gridding/Data/\$year\$/final} on the bhle4m. It
  usually gives some errors as it expect some files to be zipped,
  which aren't, but this is no problem. It also gives errors that some
  files do not exist, these are the uncertainty files which are not
  calculated for the monthly updates. Just ignore these errors.
\item Copy \texttt{$\sim$/Gridding/Data/\{\$year\}/final/*.nc.gz} to
  \texttt{$\sim$/codebase/download/ ensembles/data/months} on the
  development system bhlbontw (usually *.nc.gz only gives the files
  that are needed, but you can check it. The files needed are
  *2013\#\#.nc.gz and *2013.nc.gz.)
\item In
  \texttt{$\sim$/codebase/download/ensembles/downloadmonths.php}
  uncomment the part for the month that you did your update for by
  removing the start $<!--$ before the month is written out and the
  end $-->$ below, but above the place where the previous month is
  written out.
\item Copy the files in
  \texttt{$\sim$/codebase/download/ ensembles/data/months} to the
  operational system (same directy) and change
  \texttt{$\sim$/codebase/download/ ensembles/downloadmonths.php}
  there.
\end{enumerate}

\subsection{Preparing the plots for the climate diagnostic talk}

As of March 2013, the climate diagnostic talks are no longer given,
but plans are to merge this with the weekly weather talks on
Thursday. Therefore the method to calculate these files will still be
described below.

\begin{enumerate}
\item Go to \texttt{$\sim$/Gridding/Plot/KlimDiag/} on the bhle4m
  machine.
\item Type 'obs\_month\_temp' and give the month to plot. Press enter
  until you receive a plot on your screen. Check the plot and if it is
  okay, do the same but enter 'n' to the question if you want to show
  it on the screen. Press enter for the file name, this is the
  standard file that is used.
\item Do the same for precipitation with 'obs\_month\_rr'.
\item Do the same for precipitation with 'obs\_month\_pp'.
\item Do the same for temperature anomaly with 'means\_temp'.
\item Do the same for precipitation anomaly with 'means\_rr'.
\item Do the same for precipitation anomaly with 'means\_pp'.
\item if you need to change something, the (fixed) colour scale range
  is given at the beginning of subroutine Initializing(\dots) with the
  parameters standardmax and standardmin. Also the standarddate is
  given there, but you can change that from the command line. If you
  need to change something else, please save the original .c file to
  another name.
\item Sometimes Rob asks for a specific date. These can be produces
  with \texttt{anomaly\_rr}, \texttt{anomaly\_temp}, \texttt{obs\_rr}
  and \texttt{obs\_temp}.
\item If you get errors like \texttt{inq lon: status = 2 No such file
    or directory} it usually means that the files is zipped in
  \texttt{$\sim$/Gridding/Data/\{\$year\}/final/}. The filename is given in
  the C-code in the top subroutine.
\item If you need TX or TN, you can do a search and replace in the
  files with \texttt{\_temp.c} and use 'make' on that file. Please put
  it back to TG afterwards.
\item Run the script \texttt{convert\_and\_move.sh} to create png
  files, move the files to a directory \textit{yyyymm} and create a
  zip-file.
\item Email the zip-file to Rob Sluijter (someone else if he is away).
\end{enumerate}

\subsection{Preparing the files for the website Maandoverzicht
  Wereldweer}
\begin{enumerate}
\item To produce the plots for the website \textit{Maandoverzicht
    Wereldweer} go to
  \texttt{$\sim$/Gridding/Plot/OverzichtWereldweer} on the bhle4m
  machine.
\item Put into \texttt{getplots.sh} the files that you need and run
  it. The postscript plots are saved in \texttt{Plots}.
\item Run \texttt{converting.sh} which will convert the postscript
  plots of 2013 to png files that are needed on the website. (or do it
  manually with 'convert -rotate 90 -resize 450 $<$filename$>$.ps
  $<$filename$>$.png')
\item Copy the new .png files to
  \texttt{/data/web/www2/htdocs/klimatologie/\\maandoverzicht\_wereldweer/\{\$year\}/}
  (only reachable from your own machine)
\item Usually Else sends an email to Geert Jan to notify him that the
  files for month xxx have been copied to the directory of
  maandoverzicht wereldweer.
\end{enumerate}

\end{document}
